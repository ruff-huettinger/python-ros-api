\section*{Kurzfassung}
\addcontentsline{toc}{chapter}{Kurzfassung}
Die vorliegende Arbeit soll die Umsetzung eines Exponates für Bildungseinrichtungen mit einem mobilen Roboter als Intelligenzträger untersuchen und entwickeln. Dazu wurde im Laufe der Arbeit eine API entwickelt, das als Bindeglied zwischen der Anwendung und der Roboterkontrollarchitektur dient. Hierbei wurde das System mit Intelligenz angereichert, um so die Fähigkeiten und Klugheiten der mobilen Roboter, die heute möglich sind, spielerisch zu transportieren. So wurde mithilfe von Sensoren eine Hinderniserkennung implementiert, um vorrangig sicherheitsrelevante Fähigkeiten hardwarenah abzuwickeln, aber es wurde auch ein Ansatz für einen Algorithmus entwickelt, der ermöglicht, den Roboter aus einem Labyrinth zu befreien. Durch die offene Gestaltung der API kann die grafische Benutzeroberfläche je nach Einrichtung und Zweck angepasst werden. Das System selbst, das in dieser Arbeit untersucht und experimentell entwickelt wurde, beinhaltet viele kleinere Subsysteme, die im Einzelnen verschiedene Funktionalitäten ermöglichen. Durch diese Arbeit konnte ein Wissenslevel erarbeitet werden, um so ein System für die Firma Kurt Hüttinger GmbH \& Co. KG zu entwickeln und gegebenenfalls mit weiteren Intelligenzen zu erweitern. 
\section*{Abstract}
\addcontentsline{toc}{chapter}{Abstract}
This thesis aims to investigate and develop the implementation of an exhibit for educational institutions using a mobile robot as an intelligence carrier. For this purpose, an API was developed during the course of the work, which serves as a link between the application and the robot control architecture. In this process, intelligence was added to the system in order to playfully convey the capabilities and smarts of mobile robots that are possible today. For example, obstacle detection was implemented using sensors to handle primarily safety-related capabilities in hardware, but an approach was also developed for an algorithm that allows the robot to escape from a maze. Due to the open design of the API, the graphical user interface can be customized depending on the setup and purpose. The system itself, which was studied and experimentally developed in this work, includes many smaller subsystems that individually enable different functionalities. Through this work a level of knowledge could be worked out to develop such a system for the company Kurt Hüttinger GmbH \& Co. KG and to extend it with further intelligences if necessary.